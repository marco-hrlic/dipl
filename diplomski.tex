% Predlozak za pisanje diplomskog rada na PMF-MO
% Opcenita uputstva za LaTeX se mogu npr. naci na 
% http://web.math.hr/nastava/rp3, http://web.math.hr/nastava/s4-prof/latex.pdf
% NE PREPORUCA se "Ne baš tako kratak uvod u TEX", buduci se radi o vrlo starom prirucniku
% koji nije pogodan za moderne verzije LaTEXa.
% Originalna verzija "The not so short..." na http://tobi.oetiker.ch/lshort/lshort.pdf 
% je obnovljena i daje bolji uvid u moderne verzije LaTeXa

% Stil je optimiziran za kreiranje pdf dokumenta (npr. pomocu pdflatex-a, XeLaTeX-a)

\documentclass[a4paper,twoside,12pt]{memoir} % jednostrano: promijeniti twoside u oneside

% Paket inputenc omogucava direktno unosenje hrvatskih dijakritickih znakova 
% opcija utf8 za unicode (unix, linux, mac)
% opcija cp1250 za windowse
\usepackage[utf8]{inputenc}  % ukoliko se koristi XeLaTeX onda je \usepackage{xunicode}\usepackage{xltxtra}
\usepackage{mathrsfs} 
% Stil za diplomski, unutra je ukljucena podrska za hrvatski jezik
\usepackage{diplomski}
% bibliografija na hrvatskom
\usepackage[languagenames,fixlanguage,croatian]{babelbib} % zahtijeva datoteku croatian.bdf
% hiperlinkovi 
\usepackage[pdftex]{hyperref} % ukoliko se koristi XeLaTeX onda je \usepackage[xetex]{hyperref}
%\newcommand{\vect}[1]{\boldsymbol{\mathrm{#1}}}
\newcommand{\vect}[1]{\mathbf{#1}}
\setlength{\abovedisplayskip}{5pt}
\setlength{\belowdisplayskip}{5pt}
\setlength{\abovedisplayshortskip}{5pt}
\setlength{\belowdisplayshortskip}{5pt}

\newcommand{\card}{\text{\normalfont{card}}}
\newcommand{\supp}{\text{\normalfont{supp}}}


% Odabir familije fontova:
% koristenjem XeLaTeX-a mogu se koristiti svi fontovi instalirani na racunalu, npr
% \defaultfontfeatures{Mapping=tex-text}
% \setmainfont[Ligatures={Common}]{Hoefler Text}
% ili
% \newcommand{\nas}[1]{\fontspec{Adobe Garamond Pro}\fontsize{24pt}{24pt}\color{Chocolate}\selectfont #1}
% i onda \nas{Naslov ...}
%\usepackage{lmodern} % times new roman 
\usepackage[T1]{fontenc}
%\usepackage{newtxtext,newtxmath}
\usepackage{mathtools}

% Paket graphicx sluzi za manipuliranje grafikom 
\usepackage[pdftex]{graphicx} % ukoliko se koristi XeLaTeX onda je \usepackage[xetex]{graphicx}
% Paket amsmath je vec ukljucen
% Dodatno definirane matematicke okoline:
% teorem (okolina: thm), lema (okolina: lem), korolar (okolina: cor),
% propozicija (okolina: prop), definicija (okolina: defn), napomena (okolina: rem),
% slutnja (okolina: conj), primjer (okolina: exa), dokaz (okolina: proof)
% Definirane su naredbe za ispisivanje skupova N, Z, Q, R i C
% Definirane su naredbe za funkcije koje se u hrvatskoj notaciji oznacavaju drukcije 
% nego u americkoj: tg, ctg, ... (\tgh za tangens hiperbolni)
% Takodjer su definirane naredbe za Ker i Im (da bi se razlikovala od naredbe za imaginarni dio kompleksnog
% broja, naredba se zove \slika).

\pagestyle{headings}
% uz paket fancyhdr mogu se lako kreirati fancy zaglavlja i podnozja

% Podaci koje treba unijeti
\title{Sa\v{z}eto uzorkovanje}
\author{Marco Hrli\'c}
\advisor{Prof. dr. sc. Damir Baki\'c}  % obavezno s titulom (prof. dr. sc ili doc. dr. sc.)
\date{2019.}  % oblika mjesec, godina

% Moguce je unijeti i posvetu
% Ukoliko nema posvete, dovoljno je iskomentirati/izbrisati sljedeci redak 
\dedication{Albini}

\begin{document}

% Naredna frontmatter generira naslovnu stranicu, stranicu za potpise povjerenstva, eventualnu posvetu i sadrzaj
% Moze se iskomentirati ukoliko nije u pitanju konacna verzija
\frontmatter

% Tekst diplomskog ...

% Diplomski rad treba poceti s uvodnim poglavljem  
\begin{intro}
...
\end{intro}

\chapter[Rijetka rje\v{s}enja][Rijetka rje\v{s}enja]{Rijetka rje\v{s}enja}	
% ukoliko naslov nije jako dugacak dovoljno je samo \chapter{Naslov poglavlja} 

\section[Rijetsko i sa\v{z}etost vektora][Rijetsko i sa\v{z}etost vektora]{Rijetsko i sa\v{z}etost vektora}
%\subsection{Naslov podsekcije}
Uvedimo potrebnu notaciju. Neka je $[N]$ oznaka za skup $\{1,2,...,N\}$ gdje je $N\in\N$. Sa $\card(S)$ ozna\v{c}ujemo kardinalitet skupa $S$. Nadalje, $\bar{S}$ je komplement od $S$ u $[N]$, tj. $\bar{S}=[N]\backslash S$.

\begin{defn}
    Nosa\v{c} vektora $\vect{x} \in \C^{N}$ je skup indeksa njegovih ne-nul elemenata, tj.
    $$\supp(\vect{x}):=\{j\in[N]:x_j \neq 0 \}$$
\end{defn}

\noindent Za vektor $\vect{x}\in\C^{N}$ ka\v{z}emo da je $s$-rijedak ako vrijedi $$\|\vect{x}\|_0 := \card(\supp(\vect{x})) \leq s$$
Primjetimo,
$$\|\vect{x}\|_p^p := \sum_{j=1}^N|x_j|^p \xrightarrow{p\rightarrow 0} \sum_{j=1}^N\bold{1}_{\{x_j \neq 0\}} = \card(\{j \in [N]:x_j \neq 0\}) = \|\vect{x}\|_0$$
Gdje smo koristili da je $\bold{1}_{\{x_j \neq 0\}} = 1$  ako je $x_j \neq 0$ te $\bold{1}_{\{x_j \neq 0\}} = 0$  ako je $x_j = 0$. Drugim rije\v{c}ima, $\|\vect{x}\|_0$ je limes $p$-te potencije $\ell_p$-kvazinorme vektora $\vect{x}$ kada $p$ te\v{z}i k nuli. Kvazinorma definira se jednako kao standardna $\ell_p$-norma, jedino \v{s}to nejednakost trokuta oslabimo, tj. 
$$\|\vect{x}+\vect{y}\|\leq C(\|\vect{x}\|+\|\vect{y}\|)$$ 
za neku konstantu $C \geq 1$.
Funkciju $\|\cdot\|_0$ \v{c}esto nazivamo $\ell_0$-norma vektora $x$, iako  ona nije niti norma niti kvazinorma. U samoj praksi, te\v{s}ko je tra\v{z}iti rijetkost vektora, pa je stoga prirodno zahtjevati slabiji uvjet \textit{kompresibilnosti}.  
\begin{defn}
    $\ell_p$-gre\v{s}ku najbolje $s$-rijetke aproksimacije vektora $\vect{x}\in\C^{N}$ definiramo sa 
    $$\sigma_s(\vect{x})_p := \inf\big\{\|\vect{x}-\vect{z}\|_p,\ \vect{z} \in \C^{N} \ \small{\text{je s-rijedak}}\big\}$$
\end{defn}
\indent Primjetimo da se infimum posti\v{z}e za svaki $s$-rijedak vektor $\vect{z} \in \C^{N}$ koji ima ne-nul elemente koji su jednaki sa $s$ najve\'cih komponenti vektora $\vect{x}$. Iako takav $\vect{z} \in \C^{N}$ nije jedinstven, on posti\v{z}e infimum za svaki $p > 0$. Neformalno, mogli bi re\'ci da je vektor $\vect{x} \in \C^{N}$ \textit{kompresibilan} ako gre\v{s}ka njegove najbolje $s$-rijetke aproksimacije brzo konvergira u $s$. Da bi to formalno iskazali, od koristi \'ce biti ocjena na $\sigma_s(\cdot)_p$. Po\v{s}to nam za to ne\'ce biti va\v{z}an poredak elemenata vektora $\vect{x}$, uvodimo sljede\'cu definiciju koja \'ce nam olaksati ra\v{c}un.

\begin{defn}
    Nerastu\'ci poredak vektora $\vect{x} \in \C^{N}$ je vektor $\vect{x}^* \in \R^{\N}$ takav da
    $$x^*_1 \geq x^*_2 \geq x^*_3 \geq \dots \geq 0$$
    te postoji permutacije $\pi : [N]\rightarrow[N]$ takva da $x^*_j=|x_{\pi(j)}|$ za sve $j\in [N]$.
\end{defn}
\begin{prop}\label{osnovna_ocjena_lp_greske}
    Za svaki $q > p > 0$ i za svaki $\vect{x}\in \C^{N}$ vrijedi
    $$\sigma_s(\vect{x})_q \leq \frac{1}{s^{1/p - 1/q}}\|\vect{x}\|_p.$$
\end{prop}
\begin{proof}
    Neka je $\vect{x}^* \in \R^N$ nerastu\'ci poredak vektora $\vect{x}\in\C^N$. Tada slijedi,
    \begin{equation*}
    \begin{split} 
        \sigma_s(\vect{x})_q^q &= \sum_{j=s+1}^{N}(x_j^*)^q=\sum_{j=s+1}^{N}(x_j^*)^p(x_j^*)^{q-p} \leq (x_s^*)^{q-p} \sum_{j=s+1}^{N}(x_j^*)^p \leq \bigg(\frac{1}{s}\sum_{j=1}^{s}(x_j^*)^p\bigg)^{\frac{q-p}{p}}\bigg( \sum_{j=s+1}^N(x_j^*)^p\bigg) \\ &\leq \bigg( \frac{1}{s} \|\vect{x}\|_p^p \bigg)^{\frac{q-p}{p}}\|\vect{x}\|_p^p = \frac{1}{s^{q/p-1}}\|\vect{x}\|_p^q
    \end{split}
    \end{equation*}
    Prva nejednakost slijedi iz \v{c}injenice da je $x_j^* \leq x_s^*$ za svaki $j \geq s+1$. Druga nejednakost je tako\dj er posljedica nerasta komponenti od $\vect{x}^*$. Potenciranjem obje strane s $1/q$ slijedi tvrdnja.
\end{proof}
Primjetimo da ako je $\vect{x}$ iz jedini\v{c}ne $\ell_p$-kugle za neki mali $p>0$, onda prethodna propozicija garantira kovergenciju od $\sigma_s(\vect{x})_q$ u $s$, gdje $\ell_p$-kuglu definiramo kao
$$B_p^N := \big\{ \vect{z} \in \C^N : \|\vect{z}\|_p \leq 1\big\}$$
Vratimo se sada ocjeni iz propozicije \ref{osnovna_ocjena_lp_greske}. Sljede\'ci teorem daje najmanju konstantu $c_{p,q}$ takvu da vrijedi $\sigma_s(\vect{x})_q\leq c_{p,q}s^{-1/p+1/q}\|\vect{x}\|_p$ te zapravo predstavlja ja\v{c}u tvrdnju.
\begin{thm}
    Za svaki $q > p > 0$ i za svaki $\vect{x}\in\C^N$ vrijedi
    \begin{equation*}
    \sigma_s(\vect{x})_q \leq \frac{c_{p,q}}{s^{1/p - 1/q}}\|\vect{x}\|_p
    \end{equation*}
    gdje je
    $$
    c_{p,q} := \bigg[ \bigg(\frac{p}{q}\bigg)^{p/q}\bigg( 1-\frac{p}{q}^{1-p/q}\bigg) \bigg]^{1/p}\leq1.
    $$
\end{thm}
Istaknimo za \v{c}esti odabir $p=1$ i $q=2$
\begin{equation*}
    \sigma_s(\vect{x})_2 \leq \frac{1}{2\sqrt{s}}\|\vect{x}\|_1
\end{equation*}
\begin{proof}
    Neka je $\vect{x}^*$ nerastu\'ci poredak vektora $\vect{x}\in\C^N$ i $\alpha_j := (x_j^*)^p$. Dokazati \'cemo ekvivaltenu tvrdnju
    \begin{equation}\label{ocjena_ekv_tvrdnja}
    \begin{rcases}
{\alpha_1 \geq \alpha_2 \geq \cdots \geq \alpha_N \geq 0} \\
{\alpha_1 + \alpha_2 + \cdots + \alpha_N \leq 1} 
\end{rcases}\implies \alpha_{s+1}^{q/p} + \alpha_{s+2}^{q/p} + \cdots + \alpha_{s+N}^{q/p} \leq \frac{c^q_{q}}{s^{q/p-1}}
    \end{equation}
    Stoga, za $r:=q/p>1$, problem se svodi na maksimizaciju konveksne funkcije
    $$
    f(\alpha_1, \alpha_2, \dots, \alpha_N) := \alpha_{s+1}^r + \alpha_{s+2}^r + \cdots +\alpha_{N}^r
    $$
    na konveksnom mnogokutu
    $$
    \mathcal{C} := \big\{ (\alpha_1, \cdots, \alpha_N)\in \R^N :  \alpha_1 \geq \alpha_2 \geq \cdots \geq \alpha_N \geq 0 i  \alpha_1 + \alpha_2 + \cdots + \alpha_N \leq 1\big\}
    $$
    Prema teoremu (todo) $f$ posti\v{z}e maksimum na nekom od vrhova mnogokuta $\mathcal{C}$, a vrhovi od $\mathcal{C}$ su dani kao sjeci\v{s}ta  $N$ hiperplohi koje dobijemo tako da u \eqref{ocjena_ekv_tvrdnja} $N$ nejednakost pretvorimo u jednakost. Mogu\v{c}nosti su:
    \begin{itemize}
        \item $\alpha_1=\cdots=\alpha_N \quad\implies\quad f(\alpha_1, \alpha_2, \dots, \alpha_N) = 0$.
        \item $\alpha_1+\cdots+\alpha_N=1$ i $\alpha_1=\cdots=\alpha_k>\alpha_{k+1}=\cdots=\alpha_N=0$ za neki $1\leq k \leq s \quad \implies \quad f(\alpha_1, \alpha_2, \dots, \alpha_N) = 0$
        \item $\alpha_1+\cdots+\alpha_N=1$ i $\alpha_1=\cdots=\alpha_k>\alpha_{k+1}=\cdots=\alpha_N=0$ za neki $s+1\leq k \leq N \quad \implies \quad \alpha_1=\cdots\alpha_k=1/k$ te $f(\alpha_1, \alpha_2, \dots, \alpha_N) = (k-s)/k^r$
    \end{itemize}
    Dakle, slijedi da 
    $$
    \max\limits_{(\alpha_1,\dots,\alpha_N)\in\mathcal{C}} f(\alpha_1, \alpha_2, \dots, \alpha_N) = \max\limits_{s+1\leq k \leq N} \frac{k-s}{k^r}
    $$
    Shvatimo sada $k$ kao realnu varijablu i zamjetimo da $g(k):=(k-s)/k^r$ raste do kriti\v{c}ne to\v{c}ke $k^*=(r/(r-1))s$ nakon koje opada.
    $$
    \max\limits_{(\alpha_1,\dots,\alpha_N)\in\mathcal{C}} f(\alpha_1, \alpha_2, \dots, \alpha_N) \leq g(k^*) = \frac{1}{r}\bigg( 1- \frac{1}{r}\bigg)^{r-1}\frac{1}{s^r-1}=c^q_{p,q}\frac{1}{s^{q/p}-1}
    $$
\end{proof}
\indent Alternativni na\v{c}in na koji bi mogli definirati pojam \textit{kompresibilnosti} za vektor $\vect{x}\in\C^N$ je da zahtjevamo da je broj
$$\card(\{j\in[N]:|x_j|\geq t\})$$
njegovih zna\v{c}ajnih ne-nul komponenti dovoljno mali. Ovaj pristup vodi na definiciju slabih $\ell_p$-prostora.
\begin{defn}
Za $p>0$, slabi $\ell_p$-prostor s oznakom $w\ell_p^N$ definiramo kao prostor $\C^N$ sa kvazinormom
\begin{equation}\label{slaba_kvazinorma}
    \|x\|_{p, \infty}:=\inf\bigg\{ M \geq 0: \card (\{j\in [N]: |x_j|\geq t \})\leq \frac{M^P}{t^p},\ \forall t>0    \bigg\}
\end{equation}
\end{defn}
\noindent
Da bi pokazali da je \eqref{slaba_kvazinorma} zapravo kvazinorma, potreban nam je sljede\'ci rezultat.
\begin{prop}
    Neka su $\vect{x}^1,\dots\vect{x}^k\in\C^N$. Tada za svaki $p>0$ vrijedi 
    \begin{equation*}
    \|\vect{x}^1+\dots+\vect{x}^k\|_{p,\infty} \leq k^{\max\{1, 1/p\}}(\|\vect{x}^1\|_{p, \infty} + \cdots + \|\vect{x}^k\|_{p, \infty})
    \end{equation*}
\end{prop}
\begin{proof}
    Neka je $t>0$. Ako je $|x_j^1+\cdots+x_j^k|\geq t$ za neki $j\in [N]$, tada imamo da je $|x_j^i|\geq t/k$ za neki $i \in [k]$. Dakle, vrijedi
    \begin{equation*}
        \big\{ j\in [N]:|x_j^1+\cdots+x_j^k| \geq t \big\} \subset \bigcup\limits_{i\in [k]} \big \{ j \in [N] : |x_j^i| \geq t/k \big \}.
    \end{equation*}
\end{proof}





\nocite{*}
% Na kraju diplomkog rada stavlja se  bibliografija
% Najprije definiramo nacin prikazivanja bibliografije, u ovom slucaju verzija amsplain stila
\bibliographystyle{babamspl} % babamspl ili babplain

% U datoteku diplomski.bib se stavljaju bibliografske reference
% Bibliografske reference u bib formatu se mogu dobiti iz MathSciNet baze, Google Scholara, ArXiva, ...
\bibliography{diplomski}

\pagestyle{empty} % ne zelimo brojanje sljedecih stranica

% I na koncu idu sazeci na hrvatskom i engleskom

\begin{sazetak}
Ukratko ...
\end{sazetak}

\begin{summary}
In this ...
\end{summary}

% te zivotopis

\begin{cv}
Dana ...
\end{cv}

\end{document}
