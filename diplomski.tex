% Predlozak za pisanje diplomskog rada na PMF-MO
% Opcenita uputstva za LaTeX se mogu npr. naci na 
% http://web.math.hr/nastava/rp3, http://web.math.hr/nastava/s4-prof/latex.pdf
% NE PREPORUCA se "Ne baš tako kratak uvod u TEX", buduci se radi o vrlo starom prirucniku
% koji nije pogodan za moderne verzije LaTEXa.
% Originalna verzija "The not so short..." na http://tobi.oetiker.ch/lshort/lshort.pdf 
% je obnovljena i daje bolji uvid u moderne verzije LaTeXa

% Stil je optimiziran za kreiranje pdf dokumenta (npr. pomocu pdflatex-a, XeLaTeX-a)

\documentclass[a4paper,twoside,12pt]{memoir} % jednostrano: promijeniti twoside u oneside

% Paket inputenc omogucava direktno unosenje hrvatskih dijakritickih znakova 
% opcija utf8 za unicode (unix, linux, mac)
% opcija cp1250 za windowse
\usepackage[utf8]{inputenc}  % ukoliko se koristi XeLaTeX onda je \usepackage{xunicode}\usepackage{xltxtra}
\usepackage{mathrsfs} 
% Stil za diplomski, unutra je ukljucena podrska za hrvatski jezik
\usepackage{diplomski}
% bibliografija na hrvatskom
\usepackage[languagenames,fixlanguage,croatian]{babelbib} % zahtijeva datoteku croatian.bdf
% hiperlinkovi 
\usepackage[pdftex]{hyperref} % ukoliko se koristi XeLaTeX onda je \usepackage[xetex]{hyperref}
\usepackage{enumitem}
%\newcommand{\vect}[1]{\boldsymbol{\mathrm{#1}}}
\newcommand{\vect}[1]{\mathbf{#1}}

\newcommand{\card}{\text{\normalfont{card}}}
\newcommand{\supp}{\text{\normalfont{supp}}}
\newcommand{\norm}[1]{\|{#1}\|}
\newcommand{\rank}{\text\normalfont{rank}}


% Odabir familije fontova:
% koristenjem XeLaTeX-a mogu se koristiti svi fontovi instalirani na racunalu, npr
% \defaultfontfeatures{Mapping=tex-text}
% \setmainfont[Ligatures={Common}]{Hoefler Text}
% ili
% \newcommand{\nas}[1]{\fontspec{Adobe Garamond Pro}\fontsize{24pt}{24pt}\color{Chocolate}\selectfont #1}
% i onda \nas{Naslov ...}
%\usepackage{lmodern} % times new roman 
\usepackage[T1]{fontenc}
%\usepackage{newtxtext,newtxmath}
\usepackage{lmodern}
\usepackage{mathtools}

\usepackage{lineno}
\linenumbers

% Paket graphicx sluzi za manipuliranje grafikom 
\usepackage[pdftex]{graphicx} % ukoliko se koristi XeLaTeX onda je \usepackage[xetex]{graphicx}
% Paket amsmath je vec ukljucen
% Dodatno definirane matematicke okoline:
% teorem (okolina: thm), lema (okolina: lem), korolar (okolina: cor),
% propozicija (okolina: prop), definicija (okolina: defn), napomena (okolina: rem),
% slutnja (okolina: conj), primjer (okolina: exa), dokaz (okolina: proof)
% Definirane su naredbe za ispisivanje skupova N, Z, Q, R i C
% Definirane su naredbe za funkcije koje se u hrvatskoj notaciji oznacavaju drukcije 
% nego u americkoj: tg, ctg, ... (\tgh za tangens hiperbolni)
% Takodjer su definirane naredbe za Ker i Im (da bi se razlikovala od naredbe za imaginarni dio kompleksnog
% broja, naredba se zove \slika).

\pagestyle{headings}
% uz paket fancyhdr mogu se lako kreirati fancy zaglavlja i podnozja

% Podaci koje treba unijeti
\title{Sa\v{z}eto uzorkovanje}
\author{Marco Hrli\'c}
\advisor{Prof. dr. sc. Damir Baki\'c}  % obavezno s titulom (prof. dr. sc ili doc. dr. sc.)
\date{2019.}  % oblika mjesec, godina

% Moguce je unijeti i posvetu
% Ukoliko nema posvete, dovoljno je iskomentirati/izbrisati sljedeci redak 
\dedication{Albini}

\begin{document}
\setlength\abovedisplayskip{10pt}
\setlength\belowdisplayskip{10pt}
\setlength\abovedisplayshortskip{10pt}
\setlength\belowdisplayshortskip{10pt}

% Naredna frontmatter generira naslovnu stranicu, stranicu za potpise povjerenstva, eventualnu posvetu i sadrzaj
% Moze se iskomentirati ukoliko nije u pitanju konacna verzija
\frontmatter

% Tekst diplomskog ...

% Diplomski rad treba poceti s uvodnim poglavljem  
\begin{intro}
...
\end{intro}

\chapter[Rijetka rje\v{s}enja][Rijetka rje\v{s}enja]{Rijetka rje\v{s}enja}	
% ukoliko naslov nije jako dugacak dovoljno je samo \chapter{Naslov poglavlja} 

\section[Rijetsko i sa\v{z}etost vektora][Rijetsko i sa\v{z}etost vektora]{Rijetsko i sa\v{z}etost vektora}
%\subsection{Naslov podsekcije}
Uvedimo potrebnu notaciju. Neka je $[N]$ oznaka za skup $\{1,2,...,N\}$ gdje je $N\in\N$. Sa $\card(S)$ ozna\v{c}ujemo kardinalitet skupa $S$. Nadalje, $\bar{S}$ je komplement od $S$ u $[N]$, tj. $\bar{S}=[N]\backslash S$.

\begin{defn}
    Nosa\v{c} vektora $\vect{x} \in \C^{N}$ je skup indeksa njegovih ne-nul elemenata, tj.
    $$\supp(\vect{x}):=\{j\in[N]:x_j \neq 0 \}$$
\end{defn}

\noindent Za vektor $\vect{x}\in\C^{N}$ ka\v{z}emo da je $s$-rijedak ako vrijedi $$\|\vect{x}\|_0 := \card(\supp(\vect{x})) \leq s$$
Primjetimo,
$$\|\vect{x}\|_p^p := \sum_{j=1}^N|x_j|^p \xrightarrow{p\rightarrow 0} \sum_{j=1}^N\bold{1}_{\{x_j \neq 0\}} = \card(\{j \in [N]:x_j \neq 0\}) = \|\vect{x}\|_0$$
Gdje smo koristili da je $\bold{1}_{\{x_j \neq 0\}} = 1$  ako je $x_j \neq 0$ te $\bold{1}_{\{x_j \neq 0\}} = 0$  ako je $x_j = 0$. Drugim rije\v{c}ima, $\|\vect{x}\|_0$ je limes $p$-te potencije $\ell_p$-kvazinorme vektora $\vect{x}$ kada $p$ te\v{z}i k nuli. Kvazinorma definira se jednako kao standardna $\ell_p$-norma, jedino \v{s}to nejednakost trokuta oslabimo, tj. 
$$\|\vect{x}+\vect{y}\|\leq C(\|\vect{x}\|+\|\vect{y}\|)$$ 
za neku konstantu $C \geq 1$.
Funkciju $\|\cdot\|_0$ \v{c}esto nazivamo $\ell_0$-norma vektora $x$, iako  ona nije niti norma niti kvazinorma. U samoj praksi, te\v{s}ko je tra\v{z}iti rijetkost vektora, pa je stoga prirodno zahtjevati slabiji uvjet \textit{kompresibilnosti}.  
\begin{defn}
    $\ell_p$-gre\v{s}ku najbolje $s$-rijetke aproksimacije vektora $\vect{x}\in\C^{N}$ definiramo sa 
    $$\sigma_s(\vect{x})_p := \inf\big\{\|\vect{x}-\vect{z}\|_p,\ \vect{z} \in \C^{N} \ \small{\text{je s-rijedak}}\big\}$$
\end{defn}
\indent Primjetimo da se infimum posti\v{z}e za svaki $s$-rijedak vektor $\vect{z} \in \C^{N}$ koji ima ne-nul elemente koji su jednaki sa $s$ najve\'cih komponenti vektora $\vect{x}$. Iako takav $\vect{z} \in \C^{N}$ nije jedinstven, on posti\v{z}e infimum za svaki $p > 0$. Neformalno, mogli bi re\'ci da je vektor $\vect{x} \in \C^{N}$ \textit{kompresibilan} ako gre\v{s}ka njegove najbolje $s$-rijetke aproksimacije brzo konvergira u $s$. Da bi to formalno iskazali, od koristi \'ce biti ocjena na $\sigma_s(\cdot)_p$. Po\v{s}to nam za to ne\'ce biti va\v{z}an poredak elemenata vektora $\vect{x}$, uvodimo sljede\'cu definiciju koja \'ce nam olaksati ra\v{c}un.

\begin{defn}
    Nerastu\'ci poredak vektora $\vect{x} \in \C^{N}$ je vektor $\vect{x}^* \in \R^{\N}$ takav da
    $$x^*_1 \geq x^*_2 \geq x^*_3 \geq \dots \geq 0$$
    te postoji permutacije $\pi : [N]\rightarrow[N]$ takva da $x^*_j=|x_{\pi(j)}|$ za sve $j\in [N]$.
\end{defn}
\begin{prop}\label{osnovna_ocjena_lp_greske}
    Za svaki $q > p > 0$ i za svaki $\vect{x}\in \C^{N}$ vrijedi
    $$\sigma_s(\vect{x})_q \leq \frac{1}{s^{1/p - 1/q}}\|\vect{x}\|_p.$$
\end{prop}
\begin{proof}
    Neka je $\vect{x}^* \in \R^N$ nerastu\'ci poredak vektora $\vect{x}\in\C^N$. Tada slijedi,
    \begin{equation*}
    \begin{split} 
        \sigma_s(\vect{x})_q^q &= \sum_{j=s+1}^{N}(x_j^*)^q=\sum_{j=s+1}^{N}(x_j^*)^p(x_j^*)^{q-p} \leq (x_s^*)^{q-p} \sum_{j=s+1}^{N}(x_j^*)^p \leq \bigg(\frac{1}{s}\sum_{j=1}^{s}(x_j^*)^p\bigg)^{\frac{q-p}{p}}\bigg( \sum_{j=s+1}^N(x_j^*)^p\bigg) \\ &\leq \bigg( \frac{1}{s} \|\vect{x}\|_p^p \bigg)^{\frac{q-p}{p}}\|\vect{x}\|_p^p = \frac{1}{s^{q/p-1}}\|\vect{x}\|_p^q
    \end{split}
    \end{equation*}
    Prva nejednakost slijedi iz \v{c}injenice da je $x_j^* \leq x_s^*$ za svaki $j \geq s+1$. Druga nejednakost je tako\dj er posljedica nerasta komponenti od $\vect{x}^*$. Potenciranjem obje strane s $1/q$ slijedi tvrdnja.
\end{proof}
Primjetimo da ako je $\vect{x}$ iz jedini\v{c}ne $\ell_p$-kugle za neki mali $p>0$, onda prethodna propozicija garantira kovergenciju od $\sigma_s(\vect{x})_q$ u $s$, gdje $\ell_p$-kuglu definiramo kao
$$B_p^N := \big\{ \vect{z} \in \C^N : \|\vect{z}\|_p \leq 1\big\}$$
Vratimo se sada ocjeni iz propozicije \ref{osnovna_ocjena_lp_greske}. Sljede\'ci teorem daje najmanju konstantu $c_{p,q}$ takvu da vrijedi $\sigma_s(\vect{x})_q\leq c_{p,q}s^{-1/p+1/q}\|\vect{x}\|_p$ te zapravo predstavlja ja\v{c}u tvrdnju.
\begin{thm}
    Za svaki $q > p > 0$ i za svaki $\vect{x}\in\C^N$ vrijedi
    \begin{equation*}
    \sigma_s(\vect{x})_q \leq \frac{c_{p,q}}{s^{1/p - 1/q}}\|\vect{x}\|_p
    \end{equation*}
    gdje je
    $$
    c_{p,q} := \bigg[ \bigg(\frac{p}{q}\bigg)^{p/q}\bigg( 1-\frac{p}{q}^{1-p/q}\bigg) \bigg]^{1/p}\leq1.
    $$
\end{thm}
Istaknimo za \v{c}esti odabir $p=1$ i $q=2$
\begin{equation*}
    \sigma_s(\vect{x})_2 \leq \frac{1}{2\sqrt{s}}\|\vect{x}\|_1
\end{equation*}
\begin{proof}
    Neka je $\vect{x}^*$ nerastu\'ci poredak vektora $\vect{x}\in\C^N$ i $\alpha_j := (x_j^*)^p$. Dokazati \'cemo ekvivaltenu tvrdnju
    \begin{equation}\label{ocjena_ekv_tvrdnja}
    \begin{rcases}
{\alpha_1 \geq \alpha_2 \geq \cdots \geq \alpha_N \geq 0} \\
{\alpha_1 + \alpha_2 + \cdots + \alpha_N \leq 1} 
\end{rcases}\implies \alpha_{s+1}^{q/p} + \alpha_{s+2}^{q/p} + \cdots + \alpha_{s+N}^{q/p} \leq \frac{c^q_{q}}{s^{q/p-1}}
    \end{equation}
    Stoga, za $r:=q/p>1$, problem se svodi na maksimizaciju konveksne funkcije
    $$
    f(\alpha_1, \alpha_2, \dots, \alpha_N) := \alpha_{s+1}^r + \alpha_{s+2}^r + \cdots +\alpha_{N}^r
    $$
    na konveksnom mnogokutu
    $$
    \mathcal{C} := \big\{ (\alpha_1, \cdots, \alpha_N)\in \R^N :  \alpha_1 \geq \alpha_2 \geq \cdots \geq \alpha_N \geq 0 i  \alpha_1 + \alpha_2 + \cdots + \alpha_N \leq 1\big\}
    $$
    Prema teoremu (todo) $f$ posti\v{z}e maksimum na nekom od vrhova mnogokuta $\mathcal{C}$, a vrhovi od $\mathcal{C}$ su dani kao sjeci\v{s}ta  $N$ hiperplohi koje dobijemo tako da u \eqref{ocjena_ekv_tvrdnja} $N$ nejednakosti pretvorimo u jednakosti. Mogu\v{c}nosti su:
    \begin{enumerate}
        \item $\alpha_1=\cdots=\alpha_N \ \implies\  f(\alpha_1, \alpha_2, \dots, \alpha_N) = 0$.
        \item $\alpha_1+\cdots+\alpha_N=1$ i $\alpha_1=\cdots=\alpha_k>\alpha_{k+1}=\cdots=\alpha_N=0$ za neki \\ $1\leq k \leq s \  \implies \  f(\alpha_1, \alpha_2, \dots, \alpha_N) = 0$
        \item $\alpha_1+\cdots+\alpha_N=1$ i $\alpha_1=\cdots=\alpha_k>\alpha_{k+1}=\cdots=\alpha_N=0$ za neki\\ $s+1\leq k \leq N \  \implies \  \alpha_1=\cdots\alpha_k=1/k$ te $f(\alpha_1, \alpha_2, \dots, \alpha_N) = (k-s)/k^r$
    \end{enumerate}
    Dakle, slijedi da 
    $$
    \max\limits_{(\alpha_1,\dots,\alpha_N)\in\mathcal{C}} f(\alpha_1, \alpha_2, \dots, \alpha_N) = \max\limits_{s+1\leq k \leq N} \frac{k-s}{k^r}
    $$
    Shvatimo sada $k$ kao realnu varijablu i zamjetimo da $g(k):=(k-s)/k^r$ raste do kriti\v{c}ne to\v{c}ke $k^*=(r/(r-1))s$ nakon koje opada.
    $$
    \max\limits_{(\alpha_1,\dots,\alpha_N)\in\mathcal{C}} f(\alpha_1, \alpha_2, \dots, \alpha_N) \leq g(k^*) = \frac{1}{r}\bigg( 1- \frac{1}{r}\bigg)^{r-1}\frac{1}{s^r-1}=c^q_{p,q}\frac{1}{s^{q/p}-1}
    $$
\end{proof}
\indent Alternativni na\v{c}in na koji bi mogli definirati pojam \textit{kompresibilnosti} za vektor $\vect{x}\in\C^N$ je da zahtjevamo da je broj
$$\card(\{j\in[N]:|x_j|\geq t\})$$
tj. broj njegovih zna\v{c}ajnih ne-nul komponenti dovoljno mali. Ovaj pristup vodi na definiciju slabih $\ell_p$-prostora.
\begin{defn}
Za $p>0$, slabi $\ell_p$-prostor s oznakom $w\ell_p^N$ definiramo kao prostor $\C^N$ sa kvazinormom
\begin{equation}\label{slaba_kvazinorma}
    \|\vect{x}\|_{p, \infty}:=\inf\bigg\{ M \geq 0: \card (\{j\in [N]: |x_j|\geq t \})\leq \frac{M^P}{t^p},\ \forall t>0    \bigg\}
\end{equation}
\end{defn}
\noindent
Da bi pokazali da je \eqref{slaba_kvazinorma} zapravo kvazinorma, potreban nam je sljede\'ci rezultat.
\begin{prop}
    Neka su $\vect{x}^1,\dots\vect{x}^k\in\C^N$. Tada za svaki $p>0$ vrijedi 
    \begin{equation*}
    \|\vect{x}^1+\dots+\vect{x}^k\|_{p,\infty} \leq k^{\max\{1, 1/p\}}(\|\vect{x}^1\|_{p, \infty} + \cdots + \|\vect{x}^k\|_{p, \infty})
    \end{equation*}
\end{prop}
\begin{proof}
    Neka je $t>0$. Ako je $|x_j^1+\cdots+x_j^k|\geq t$ za neki $j\in [N]$, tada imamo da je $|x_j^i|\geq t/k$ za neki $i \in [k]$. Dakle, vrijedi
    \begin{equation*}
        \big\{ j\in [N]:|x_j^1+\cdots+x_j^k| \geq t \big\} \subset \bigcup\limits_{i\in [k]} \big \{ j \in [N] : |x_j^i| \geq t/k \big \}
    \end{equation*}
    pa je stoga
    \begin{align*}
        \card\big( \big\{ j\in [N] : |x_j^1+\cdots+k_j^k| \geq t \big\}\big)&\leq\sum\limits_{i\in [k]}\frac{\|\vect{x}^i\|^p_{p, \infty}}{(t/k)^p} \\ 
                                                                            &= \frac{k^p(\|\vect{x}^1\|^p_{p, \infty}+\cdots + \|\vect{x}^k\|^p_{p, \infty})}{t^p}
    \end{align*}
    Prema definiciji slabe $\ell_p$-kvazinorme \eqref{slaba_kvazinorma} vektora $\vect{x}^1+\cdots+\vect{x}^k$ dobivamo
    \begin{equation*}
        \|\vect{x}^1+\cdots+\vect{x}^k\|_{p, \infty}\leq k\big(\|\vect{x}^1\|^p_{p,\infty}+ \cdots +\|\vect{x}^k\|^p_{p,\infty}\big) 
    \end{equation*}
    Ako je $p \leq 1$, uspore\dj uju\'ci $\ell_p$ i $\ell_1$ norme na $\R^k$ slijedi
    \begin{equation*}
        \big(\|\vect{x}^1\|^p_{p,\infty}+ \cdots +\|\vect{x}^k\|^p_{p,\infty}\big)^{1/p} \leq k^{1/p-1}\big(\|\vect{x}^1\|_{p,\infty}+ \cdots +\|\vect{x}^k\|_{p,\infty}\big)
    \end{equation*}
    te ako je $p \geq 1$ slijedi
    \begin{equation*}
        \big(\|\vect{x}^1\|^p_{p,\infty}+ \cdots +\|\vect{x}^k\|^p_{p,\infty}\big)^{1/p} \leq \|\vect{x}^1\|_{p,\infty}+ \cdots +\|\vect{x}^k\|_{p,\infty}.
    \end{equation*}
    Tvrdnja slijedi kombiniranjem dobivenih ocjena.
\end{proof}

\noindent
Uzmimo $\vect{x}, \vect{y} \in \C^N$ i neka je $\lambda \in \C$ proizvoljan.

\begin{enumerate}
    \item Neka je $\|\vect{x}\|_{p, \infty}=0$. Iz \eqref{slaba_kvazinorma} slijedi $ \card(\{j \in [N]: |x_j| \geq t\}) = 0$ za svaki $t > 0$ pa je stoga broj ne-nul komponenti on $\vect{x}$ jednak nuli, tj. $\vect{x}=0$ 
    \item Ako je $\lambda$ nula, $\|\lambda \vect{x}\| = | \lambda | \| \vect{x} \|$ vrijedi trivijalno. Za $\lambda \neq 0$, imamo \\
        $\card(\{ j \in [N]: |\alpha x_j| \geq t \}) = \card(\{ j \in [N]: |x_j| \geq t/|\alpha| \})\leq (\alpha M)^p/t^p$ za svaki $t>0$. Dakle, opet $\|\lambda \vect{x}\| = | \lambda | \| \vect{x} \|$.
    \item $\|\vect{x}+\vect{y}\|\leq C(\|\vect{x}\|+\|\vect{y}\|)$ je sada direktna posljedica prethodne propozicije.
\end{enumerate}

\noindent Sljede\'ca propozicija daje alternativni izraz za slabu $\ell_p$-kvazinormu.
\begin{prop}\label{slaba_kvazinorma_2}
    Za $p>0$, vrijedi
    \begin{equation*}
        \|\vect{x}\|_{p, \infty} = \max \limits_{k \in [N]}k^{1/p}x_k^{*}
    \end{equation*}
    gdje je $\vect{x}^* \in \R^N$ nerastu\'ci poredak vektora $\vect{x}\in \C^N$.
\end{prop}
\begin{proof}
    Primjetimo prvo da iz \eqref{slaba_kvazinorma} slijedi da je $\|\vect{x}\|_{p, \infty}=\|\vect{x}^*\|_{p, \infty}$, pa zapravo pokazujemo da je $\|\vect{x}\|:= \max_{k \in [N]}k^{1/p}x_k^* = \|\vect{x}^*\|$. Nadalje, za $t>0$ vrijedi da je $\{j \in [N]: x^*_j \geq t\}=[k]$ za neki $k \in [N]$ ili je $\{j \in [N]: x^*_j \geq t\}=\emptyset$. U prvom slu\v{c}aju $t \leq x^*_k \leq \|\vect{x}\|/k^{1/p}$ pa je $\card(\{j \in [N]:x_j^* \geq t \}) = k \leq \|\vect{x}\|/k^{1/p}$. U drugom slu\v{c}aju ista nejednakost vrijedi trivijalno. Iz definicije slabe $\ell_p$-kvazinorme \eqref{slaba_kvazinorma} sada dobivamo $\|\vect{x}^*\|_{p, \infty} \leq \|\vect{x}\|$. Pretpostavimo da je $\|\vect{x}^*\|_{p, \infty} < \|\vect{x}\|$. Tada postoji $\varepsilon > 0$ takav da $(1+ \varepsilon)\|\vect{x}^*\|_{p, \infty} \leq \|\vect{x}\|$. Slijedi da je $(1 + \varepsilon)\|\vect{x}^*\| \leq  k^{1/p}x^*_k$ za neki $k \in [N]$ pa stoga
    \begin{equation*}
        [k] \subseteq \big\{ j \in [N] : (1 + \varepsilon)\|\vect{x}^*\|_{p, \infty}/k^{1/p} \leq x_j^* \big\}
    \end{equation*}
    Ponovo iz \eqref{slaba_kvazinorma} imamo
    \begin{equation*}
        k \leq \frac{\|\vect{x}^*\|^p_{p, \infty}}{\big( (1 + \varepsilon)\|\vect{x}^*\|_{p, \infty}k^{1/p}\big)^p}=\frac{k}{(1 + \varepsilon)^p}
    \end{equation*}
    Kontradikcija, dakle mora vrijediti $\|\vect{x}\| = \|\vect{x}^*\|_{p, \infty}$.
\end{proof}
\noindent Sada lagano mo\v{z}emo usporediti slabi i jaku $\ell_p$ normu,
\begin{prop}
    Za svaki $p > 0$ i za svaki $\vect{x} \in \C^N$,
    \begin{equation*}
        \|\vect{x}\|_{p, \infty} \leq \|\vect{x}\|_p
    \end{equation*}
\end{prop}
\begin{proof}
    Neka je $k \in [N]$,
    \begin{equation*}
        \|\vect{x}\|_p^p = \sum_{j=1}^{N}(x_j^*)^p \geq \sum_{j=1}^{k}(x_j^*)^p \geq k(x_k^*)^p
    \end{equation*}
    Tvrdnja slijedi potenciranjem na $1/p$ i uzimaju\'ci maksimum po $k$ i primjenom prethodne propozicije.
\end{proof}
Koriste\'ci propoziciju \eqref{slaba_kvazinorma_2} mo\v{z}emo dobiti verziju ocjene iz propozicije \eqref{osnovna_ocjena_lp_greske} sa slabom $\ell_p$ normom.
\begin{prop}
    Za svaki $q>p>0$ i $\vect{x} \in \C^N$, vrijedi
    \begin{equation*}
        \sigma_s(\vect{x})_q \leq \frac{d_{p,q}}{s^{1/p-1/q}}\|\vect{x}\|_{p, \infty}
    \end{equation*}
    gdje je
    \begin{equation*}
        d_{p,q} := \big( \frac{p}{q-p} \big)^{1/q}.
    \end{equation*}
\end{prop}
\begin{proof}
    Bez smanjenja op\v{c}enitosti mo\v{z}emo pretpostaviti da je $\norm{\vect{x}}_{p,\infty} \leq 1$, pa je $x_k^* \leq 1/k^{1/p}$ za svaki $k \in [N]$. Tada vrijedi,
    \begin{equation*}
    \sigma_s(\vect{x})^q_q = \sum_{k=s+1}^{N} (x_k^*)^q \leq \sum_{k=s+1}^N \frac{1}{k^{q/p}} \leq \int_s^N \frac{1}{t^{q/p}} dt = - \frac{1}{q/p-1} \frac{1}{t^{q/p-1}}\bigg\rvert^{t=N}_{t=s} \leq \frac{p}{q-p} \frac{1}{s^{q/p-1}}.
    \end{equation*}
    Potenciranjem sa $1/q$ slijedi tvrdnja.
\end{proof}
Prethodna propozicija daje da su vektori $\vect{x} \in \C^N$ koji su kompresibilni u smislu $\norm{\vect{x}}_{p, \infty} \leq 1$ za mali $p>0$, tako\dj er kompresibilni u smislu da gre\v{s}ka njihove najbolje $s$-rijetke aproksimacije brzo konvergira sa $s$. Iska\v{z}imo jo\v{s} jedan tehni\v{c}ki rezultat,
\begin{lem}
    Neka su $\vect{x}, \vect{y} \in \C^N$. Tada vrijedi,
    \begin{equation} \label{nerastuci_poredak_ocjena_1}
        \norm{\vect{x}^* - \vect{y}^*}_{\infty} \leq \norm{\vect{x} - \vect{y}}_{\infty}
    \end{equation}
    Nadalje, za $s \in [N]$,
    \begin{equation}\label{nerastuci_poredak_ocjena_2}
        |\sigma_s(\vect{x})_1 - \sigma(\vect{y})_1| \leq \norm{\vect{x} - \vect{y}}_1
    \end{equation}
    i za $k>s$,
    \begin{equation}\label{nerastuci_poredak_ocjena_3}
        (k-s)x_k^* \leq \norm{\vect{x} - \vect{y}}_1 + \sigma_s(\vect{y})_1
    \end{equation}
\end{lem}
\begin{proof}
    Za $j \in [N]$, skup indeksa $j$ najve\'cih komponenti vektora $\vect{x}$ ima ne-trivijalni presjek sa skupom od $N-j+1$ najmanjih komponenti vektora $\vect{y}$. Izaberimo indeks $l$ iz tog presjeka. Tada vrijedi, 
    \begin{equation*}
        x_j^* \leq |x_l| \leq |y_l| + \norm{\vect{x} - \vect{y}}_{\infty} \leq z_j^* + \norm{\vect{x} - \vect{y}}_{\infty}
    \end{equation*}
    Zamjenom uloga od $\vect{x}$ i $\vect{y}$ slijedi \eqref{nerastuci_poredak_ocjena_1}.
    Neka je $\vect{v} \in \C^N$ najbolja $s$-rijetka aproksimacija vektora $\vect{y}$. Tada
    \begin{equation*}
        \sigma_s(\vect{x})_1 \leq \norm{\vect{x} - \vect{v}}_1 \leq \norm{\vect{x} - \vect{y}}_1 + \norm{\vect{y} - \vect{v}}_1 = \norm{\vect{x} - \vect{y}}_1 + \sigma_s(\vect{y})_1 
    \end{equation*}
    Ponovno, zbog simetrije slijedi \eqref{nerastuci_poredak_ocjena_2}. Napokon, ocjena \eqref{nerastuci_poredak_ocjena_3} slijedi iz \eqref{nerastuci_poredak_ocjena_2} te iz \v{c}injenice
    \begin{equation*}
        (k-s)x_k^* \leq \sum_{j=s+1}^{k}x_j^* \leq \sum_{j \geq s+1} x_j^* = \sigma_s(\vect{x})_1.
    \end{equation*}
\end{proof}

\section[Minimalni broj mjerenja][Minimalni broj mjerenja]{Minimalni broj mjerenja}
Problem sa\v{z}etog uzorkovanja sastoji se od rekonstrukcije $s$-rijetkog vektora $\vect{x} \in \C^N$ iz sustava
$$\vect{y} = \vect{A}\vect{x}$$
Matricu $\vect{A} \in \C^{m\times N}$ nazivamo \textit{matrica mjerenja}. Ako je $m < N$, za ovakav sustav linearnih jednad\v{z}bi ka\v{z}emo da je \textit{neodre\dj en}. Iako iz klasi\v{c}ne teorije linearne algebre ovakvi sustavi imaju beskona\v{c}no mnogo rije\v{s}enja, pokazati \'ce se da je dodatna pretpostavka rijetkosti vektora $x$ dovoljno za jedinstvenost rje\v{s}enja. U ovom poglavlju istra\v{z}iti \'cemo koji je minimalni broj mjerenja, tj. $m$ broj redaka matrice $\vect{A}$, koji garantira rekonstrukciju $s$-rijetkog vektora $\vect{x}$. Zapravo, postoje dva pristupa ovom problemu. Mo\v{z}emo zahtjevati da problem mjerenja rekonstruira sve $s$-rijetke vektore $\vect{x} \in \C^N$ istodobno ili mo\v{z}emo tra\v{z}iti rekonstrukciju specifi\v{c}nog, tj. predodre\dj enog vektora $\vect{x} \in \C^N$. Taj pristup \v{c}ini se neprirodan, no pokazuje se da je on va\v{z}an u prou\v{c}avanju problema gdje matricu $\vect{A}$ biramo nasumi\v{c}no. \\ 
\indent Poka\v{z}imo da su za danu rijetkost $s$, matricu $\vect{A} \in \C^{m \times N}$ i $s$-rijedak vektor $\vect{x} \in \C^N$, naredne tvrdnje ekvivaltentne: 
\begin{enumerate}
    \item Vektor $\vect{x}$ je jedinstveno $s$-rijetko rje\v{s}enje sustava $\vect{A}\vect{z}=\vect{y}$ gdje je $\vect{y} = \vect{Ax}$, tj. $\{\vect{z} \in \C^N : \vect{A}\vect{z}= \vect{A}\vect{x},\ \norm{\vect{z}}_0 \leq s\} = \{\vect{x}\}$
    \item Vektor $\vect{x}$ je jedinstveno rje\v{s}enje problema minimizacije
        \begin{equation}
            \min\limits_{\vect{z} \in \C^N} \norm{\vect{z}}_0\quad \text{uz uvjet}\ \vect{Az} = \vect y \label{problem_minimizacije} 
        \end{equation}
\end{enumerate}
Ako je $\vect{x} \in \C^N$ jedinstveno $s$-rijetko rje\v{s}enje od $\vect{Az} = \vect y$ takvo da je $\vect y = \vect{Ax}$, onda rje\v{s}enje $x^{\sharp}$ od \eqref{problem_minimizacije} je $s$-rijetko i zadovoljava $\vect{Ax} = \vect y$ pa je $\vect x^\sharp = \vect x$. Drugi smjer slijedi trivijalno.


\subsection[Rekonstrukcija svih rijetkih vektora][Rekonstrukcija svih rijetkih vektora]{Rekonstrukcija svih rijetkih vektora}
Neka je $\vect{A} \in \C^{m \times N}$ i $S \subset [N]$, sa $\vect A_S$ ozna\v{c}ujemo matricu formiranu od stupaca od $\vect A$ indeksiranih sa $S$. Sli\v{c}no, sa $\vect x_S$ ozna\v{c}ujemo ili vektor iz $\C^{S}$ koji se sastoji od komponenti vektora $\vect x$ indeksiranih po $S$, tj. $(\vect x_S)_l = x_l$ za sve $l \in S$, ili vektor iz $\C^N$ koji se podudara s $\vect x$ na komponentama indeksiranim u $S$ i jednak je nula na indeksima koji nisu u $S$, tj. $(\vect x_S)_l = x_l$ za $l \in S$ i $(\vect x_S)_l =0$ za $ l \notin S$. Iz konteksta \'ce uvijek biti jasno na koju definiciju se misli.

\begin{thm} \label{rekonstrukcija_tm1}
    Neka je $\vect A \in \C^{m \times N}$. Ekvivalentno je:
    \begin{enumerate}[label=(\alph*)]
        \item Svaki $s$-rijedak vektor $\vect x \in \C^N$ je jedinstveno rje\v{s}enje od $\vect{Ax}=\vect{Az}$, tj. ako je $\vect{Ax}=\vect{Az}$ i ako su $\vect x$, $\vect z$ oboje $s$-rijetki tada $\vect x = \vect z$.
        \item Jezgra od $\vect A$ ne sadr\v{z}i niti jedan $2s$-rijedak vektor osim nul-vektora, tj. $\ker \vect A \cap \{\vect z \in \C^N: \norm{\vect z}_0 \leq 2s\} = \{\vect 0\}$
        \item Za svaki $S \subset [N]$ takav da $\card(S) \leq 2s$, podmatrica $\vect A_S$ je injektivna kao preslikavanje sa $\C^S$ u $\C^m$.
        \item Svaki skup od $2s$ stupaca matrice $\vect A$ je linearno nezavisan skup.
    \end{enumerate}
\end{thm}
\begin{proof}
    \begin{itemize}
        \item[]$(b)\implies(a)$. Neka su $\vect x$ i $\vect z$ $s$-rijetki vektori takvi da $\vect{Ax} = \vect{Az}$. Tada je $\vect x - \vect z$ $2s$-rijedak i $\vect A(\vect x - \vect z) = \vect 0$. Po\v{s}to $\ker \vect A$ ne sadr\v{z}i $2s$-rijetke vektore osim nul-vektora, mora vrijediti $\vect x = \vect z$. \\
        \item[] $(a)\implies(b)$. Obratno, pretpostavimo da za svaki $s$-rijetki vektor $\vect x \in \C^N$ vrijedi $\{\vect z \in \C^N : \vect{Az} = \vect{Ax}, \norm{\vect z}_0 \leq s\} = \{\vect x\}$. Neka je $\vect v \in \ker \vect A$, $2s$-rijedak. Tada $\vect v$ mo\v{z}emo rastaviti kao $\vect v = \vect x - \vect z$ gdje su $\vect x$ i $\vect z$ $s$-rijetki takvi da $\supp(\vect x)\cap \supp(\vect z) = \emptyset$. Imamo da je $\vect{Ax}=\vect{Az}$ pa prema pretpostavci vrijedi $\vect{x}=\vect{z}$. Po\v{s}to su nosa\v{c}i od $\vect x$ i $\vect z$ disjunktni, mora vrijediti $\vect x = \vect z = \vect 0$ pa je stoga i $\vect v = 0$.\\
        \item[] $(b)\implies(c)$. Pretpostavimo suprotno, $\ker \vect A \cap \{\vect z \in \C^N: \norm{\vect z}_0 \leq 2s\} = \{\vect 0\}$ i da postoji $S \in [N]$ takav da je $\card(S) \leq 2s$ te da $\vect A_s$ nije injektivna. To zna\v{c}i da postoji vektor $\vect x \in \C^{\card(S)} \backslash \{\vect 0\}$ takav da je $\vect A_S \vect x = \vect 0$. Definiramo vektor $\tilde{\vect{x}}\in \C^N$ sa 
            \begin{equation*}
                \tilde{x}_j = 
                \begin{cases}
                    x_j \quad & \text{za}\ j \in S \\
                    0 \quad & \text{za}\  j \in \bar S \\
                \end{cases}
            \end{equation*}
            Dakle, imamo $\vect x \neq \vect 0$, $\norm{\vect x}_0 \leq 2s$ i vrijedi $\vect{Ax}=0$, tj. $\vect x \in \ker \vect A$. Kontradikcija s $(b)$.
        \item[]$(c)\implies(d)$. Odaberimo $2s$  stupaca od $\vect A$. Skup indeksa tih stupaca ozna\v{c}imo sa $S$. Prema $(c)$, matrica $\vect A_S$ je injektivna, a to zna\v{c}i da su njeni stupci linearno nezavisni, pa su stoga i $2s$ odabranih stupaca matrice $\vect A$ linearno nezavisni.
        \item[]$(d)\implies(b)$. Pretpostavimo da jezgra od $\vect A$ sadr\v{z}i $2s$-rijedak ne-nul vektor $\vect x \in \C^N$. Neka je $S$ skup indeksa ne-nul elemenata vektora $\vect x$. To zna\v{c}i da je $\vect A_S \vect x_S = 0$, i $\vect x_S \neq \vect 0$. Dakle $\vect A_S$ nije injektivna, pa stoga i skup stupaca od $\vect A$ indeksiranih sa $S$ nije linearno nezavisan, \v{s}to je kontradikcija sa $(d)$.
\end{itemize}
\end{proof}

Uo\v{c}imo da ako je mogu\v{c}e rekonstruirati svaki $s$-rijedak vektor $\vect x \in \C^N$ iz vektora mjerenja $\vect y = \vect{Ax} \in \C^m$, tada vrijedi $(a)$. Prema pro\v{s}lom teoremu tada vrijedi i tvrdnja $(d)$ pa je stoga $\rank(\vect A) \geq 2s$. Tako\dj er vrijedi da je $\rank(\vect A) \leq m$ pa imamo 
\begin{equation*}
    m \geq 2s.    
\end{equation*}
To zna\v{c}i da je potrebno barem $2s$ mjerenja da bi rekonstruirali svaki $s$-rijedak vektor. Pokazati \'cemo da je, makar u teoriji, dovoljno to\v{c}no $2s$ mjerenja.

\begin{thm}
    Za svaki $N \geq 2s$, postoji matrica mjerenja $\vect A \in \C^{2s \times N}$ takva da se svaki $s$-rijedak vektor $\vect x \in \C^N$ mo\v{z}e rekonstruirati iz vektora mjerenja $\vect y = \vect{Ax} \in \C^m$ kao rje\v{s}enje problema minimizacije \eqref{problem_minimizacije}.
\end{thm}
\begin{proof}
    Fiksirajmo $t_N>\cdots t_2 > t_1 > 0$ i neka je $\vect A \in \C^{2s \times N}$ dana sa
    \begin{equation}\label{vandermont_matrica}
        \vect A = 
        \begin{bmatrix}
            1 & 1 & \cdots & 1 \\ 
            t_1 & t_2 & \cdots & t_N \\
            \vdots & \vdots & \cdots & \vdots \\
            t_1^{2s-1} & t_2^{2s-1} & \cdots & t_N^{2s-1} \\
        \end{bmatrix}
    \end{equation}
    Nadalje, neka je $S=\{j_1 < \cdots < j_{2s}\}$ skup indeksa. Matrica $\vect A_S \in \C^{2s \times 2s}$ je transponirana \textit{Vandermontova matrica}. Prema (TODO) slijedi
    \begin{equation*}
        \det(\vect{A}_S) = \prod_{k < l} (t_{j_l} - t_{j_k})>0.
    \end{equation*}
    To zna\v{c}i da je matrica $\vect A$ invertibilna, pa posebno i injektivna. Tada je zadovoljena tvrdnja $(c)$ teorema \eqref{rekonstrukcija_tm1}, pa je po istom teoremu zadovoljena i tvrdnja $(a)$, tj. svaki $s$-rijedak vektor $\vect x \in \C^N$ zadovoljava $\vect{Az}=\vect{Ax}$. Stoga je taj vektor mogu\'ce jedinstveno rekonstruirati putem minimizacije \eqref{problem_minimizacije}.
\end{proof}
\indent Zapravo, mnogo matrica zadovoljava uvjet $(c)$ iz teorema \eqref{rekonstrukcija_tm1}. Na primjer, potencije od $t_1,\dots,t_N$ u \eqref{vandermont_matrica} ne moraju biti uzastopne. Nadalje, brojevi $t_1,\dots,t_N$ ne moraju biti pozitivni, niti realni sve dok vrijedi $\det(\vect A_S) \neq 0$. Posebno, mo\v{z}emo uzeti $t_l = e^{2\pi i (l-1)/N}$ za $l \in [N]$, teorem (TODO) garantira da parcijalna Fourierova matrica
\begin{equation*}
   \vect A = 
   \begin{bmatrix*}
       1 & 1 & 1 & \cdots & 1 \\
       1 & e^{2 \pi i/ N} & e^{2 \pi i2/ N} & \cdots & e^{2 \pi i(N-1)/ N} \\ 
       \vdots & \vdots & \vdots & \vdots & \vdots \\ 
       1 & e^{2 \pi i(2s-1)/ N} & e^{2 \pi i(2s-1)2/ N} & \cdots & e^{2 \pi i(2s-1)(N-1)/ N} \\ 
   \end{bmatrix*}
\end{equation*}
rekonstruira svaki $s$-rijedak vektor $\vect x \in \C^N$ iz $\vect y = \vect{Ax} \in \C^{2s}$.
Zapravo mo\v{z}e se pokazati da skup $(2s) \times N$ matrica takvih da $\det(\vect A_S) = 0$ za neki $S \subset [N]$ i $\card(S) \leq 2s$ ima Lebesgueovu mjeru nula, pa stoga gotovo sve $(2s) \times N$ matrice rekonstruiraju svaki $s$-rijedak vektor $\vect x \in \C^N$ iz $\vect y = \vect{Ax} \in \C^{2s}$. Me\dj utim u praksi nije isplativo rje\v{s}avati problem minimizacije \eqref{problem_minimizacije}, \v{s}to \'cemo kasnije i pokazati.




\subsection[Rekonstrukcija zadanog rijetkog vektora][Rekonstrukcija zadanog rijetkog vektora]{Rekonstrukcija zadanog rijetkog vektora}
Promatramo problem gdje je $s$-rijedak vektor $\vect x \in \C^N$ unaprijed zadan i poznat, a matricu $\vect A \in \C^{m \times N}$ \v{z}elimo odabrati tako da ona garantira rekonstrukciju vektora $\vect x$ iz mjerenja $\vect y = \vect{Ax} \in \C^m$. Isprva, ovaka pristup izgleda neprirodan zbog \v{c}injenice da je vektor $\vect x$ apriorno poznat. Ideja je da \'ce uvjeti rekonstrukcije vrijediti za gotovo sve $(s+1) \times N$ matrice, \v{s}to podupire \v{c}injenicu da se u praksi matrice mjerenja \v{c}esto odabiru na nasumi\v{c}an na\v{c}in.
\begin{thm}
    Za svaki $N \geq s + 1$ i za dani $s$-rijedak vektor $\vect x \in \C^N$, postoji matrica mjerenja $\vect A \in \C^{(s+1) \times N}$, takva da se vektor $\vect x$ mo\v{z}e rekonstruirati iz mjerenja $\vect y = \vect{Ax} \in \C^m$ kao rje\v{s}enje minimizacije \eqref{problem_minimizacije}.
\end{thm}
\begin{proof}
    
\end{proof}



\nocite{*}
% Na kraju diplomkog rada stavlja se  bibliografija
% Najprije definiramo nacin prikazivanja bibliografije, u ovom slucaju verzija amsplain stila
\bibliographystyle{babamspl} % babamspl ili babplain

% U datoteku diplomski.bib se stavljaju bibliografske reference
% Bibliografske reference u bib formatu se mogu dobiti iz MathSciNet baze, Google Scholara, ArXiva, ...
\bibliography{diplomski}

\pagestyle{empty} % ne zelimo brojanje sljedecih stranica

% I na koncu idu sazeci na hrvatskom i engleskom

\begin{sazetak}
Ukratko ...
\end{sazetak}

\begin{summary}
In this ...
\end{summary}

% te zivotopis

\begin{cv}
Dana ...
\end{cv}

\end{document}
